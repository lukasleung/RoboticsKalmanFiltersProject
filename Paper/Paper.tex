\documentclass[12pt]{article}

%----------------------------------------------------------------------------------------
%	SETTINGS
%----------------------------------------------------------------------------------------

\usepackage{amsmath}

\usepackage{float}
\usepackage{graphicx}
\usepackage[normal]{caption}
\usepackage{subcaption}
\usepackage{mathtools}

\setlength\parindent{24pt}

% Make numbering in the enumerate environment by letter rather than number (e.g. section 6)
\renewcommand{\labelenumi}{\alph{enumi}.}

%----------------------------------------------------------------------------------------
%	DOCUMENT INFORMATION
%----------------------------------------------------------------------------------------

\title{Robotics Project \\ Kalman Filter} % Title

\author{Hoang, Lukas \textsc{Leung}, Zhuoming \textsc{Tan}} % Author name

\date{\today} % Date for the report

\begin{document}

	\maketitle % Insert the title, author and date

	\begin{center}
		\begin{tabular}{l r}
			Instructor: & Professor Ken Basye \\ % Instructor/supervisor
		\end{tabular}
	\end{center}

	% If you wish to include an abstract, uncomment the lines below
	\begin{abstract}
	Kalman filter.
	\end{abstract}

	%----------------------------------------------------------------------------------------
	%	TABLE OF CONTENTS
	%----------------------------------------------------------------------------------------

	%\newpage
	%\tableofcontents
	%\newpage

	%----------------------------------------------------------------------------------------
	%	SECTION 1
	%----------------------------------------------------------------------------------------

	\section{Introduction}

	In this project we will use Kalman filter to estimate the motion path of an aircraft.

	\subsection{Physics model}\label{sub:model}
	The following model defines the signal value.
	\begin{equation}
		x_t=Ax_{t-1}+Bu_t+w_{t-1}
	\end{equation}
	Which is
	\begin{equation}
		\begin{bmatrix}
			p^{(t)}_x \\ p^{(t)}_y \\ v^{(t)}_x \\ v^{(t)}_y
		\end{bmatrix}=
		\underbrace{\begin{bmatrix}
			1 & 0 & t & 0 \\
			0 & 1 & 0 & t \\
			0 & 0 & 1 & 0 \\
			0 & 0 & 0 & 1
		\end{bmatrix}}_{\text{state transition}}
		\begin{bmatrix}
			p^{(t-1)}_x \\ p^{(t-1)}_y \\ v^{(t-1)}_x \\ v^{(t-1)}_y
		\end{bmatrix}+
		\underbrace{\begin{bmatrix}
			t^2/2 & 0 \\
			0 & t^2/2 \\
			t & 0 \\
			0 & t
		\end{bmatrix}}_{\text{control matrix}}
		\begin{bmatrix}
			a^{(t-1)}_x \\ a^{(t-1)}_y
		\end{bmatrix}+w_{t-1}
	\end{equation}


	%----------------------------------------------------------------------------------------
	%	SECTION 2
	%----------------------------------------------------------------------------------------

	\section{Conclusion}
\end{document}
