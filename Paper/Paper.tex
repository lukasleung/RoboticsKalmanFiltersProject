\documentclass[12pt]{article}

%----------------------------------------------------------------------------------------
%	SETTINGS
%----------------------------------------------------------------------------------------

\usepackage{amsmath}

\usepackage{float}
\usepackage{graphicx}
\usepackage[normal]{caption}
\usepackage{subcaption}
\usepackage{mathtools}

\setlength\parindent{24pt}

% Make numbering in the enumerate environment by letter rather than number (e.g. section 6)
\renewcommand{\labelenumi}{\alph{enumi}.}

%----------------------------------------------------------------------------------------
%	DOCUMENT INFORMATION
%----------------------------------------------------------------------------------------

\title{Robotics Project \\ Kalman Filter} % Title

\author{Hoang \textsc{Viet Nguyen}, Lukas \textsc{Leung}, Zhuoming \textsc{Tan}} % Author name

\date{\today} % Date for the report

\begin{document}

	\maketitle % Insert the title, author and date

	\begin{center}
		\begin{tabular}{l r}
			Instructor: & Professor Ken Basye \\ % Instructor/supervisor
		\end{tabular}
	\end{center}

	% If you wish to include an abstract, uncomment the lines below
	% \begin{abstract}
	% Kalman filter.
	% \end{abstract}

	%----------------------------------------------------------------------------------------
	%	TABLE OF CONTENTS
	%----------------------------------------------------------------------------------------

	%\newpage
	%\tableofcontents
	%\newpage

	%----------------------------------------------------------------------------------------
	%	SECTION 1
	%----------------------------------------------------------------------------------------

	\section{Introduction}\label{sec:intro}

	In this project we will use Kalman filter to estimate the motion path of an aircraft. We are writing our program in python.

	\subsection{Physics model}\label{sub:model}
	The following model defines the state estimation:
	\begin{equation}
		\hat{\mathbf{x}}_t=\mathbf{A}\hat{\mathbf{x}}_{t-1}+\mathbf{B}\mathbf{u}_t+\mathbf{w}_{t-1}
	\end{equation}
	Which is
	\begin{equation}
		\begin{bmatrix}
			p^{(t)}_x \\ p^{(t)}_y \\ v^{(t)}_x \\ v^{(t)}_y
		\end{bmatrix}=
		\underbrace{\begin{bmatrix}
			1 & 0 & t & 0 \\
			0 & 1 & 0 & t \\
			0 & 0 & 1 & 0 \\
			0 & 0 & 0 & 1
		\end{bmatrix}}_{\text{state transition}}
		\begin{bmatrix}
			p^{(t-1)}_x \\ p^{(t-1)}_y \\ v^{(t-1)}_x \\ v^{(t-1)}_y
		\end{bmatrix}+
		\underbrace{\begin{bmatrix}
			t^2/2 & 0 \\
			0 & t^2/2 \\
			t & 0 \\
			0 & t
		\end{bmatrix}}_{\text{control matrix}}
		\begin{bmatrix}
			a^{(t-1)}_x \\ a^{(t-1)}_y
		\end{bmatrix}+\text{noise}
	\end{equation}
	whose $\text{noise}\sim\mathcal{N}\left(0,\Sigma_x\right)$. The state transition matrix $A$ and control matrix $B$ are what we use in the program. They are only dependent on $t$, here representing the time step, so are constant matricies if we run the program with a fixed time interval setting. And the following defines the observation from the state
	\begin{equation}
		\mathbf{z}_k=\mathbf{H}\hat{\mathbf{x}}_k+\text{noise}
	\end{equation}
	which is
	\begin{equation}
		\begin{bmatrix}
			p^{(t)}_x \\ p^{(t)}_y
		\end{bmatrix}=
		\underbrace{\begin{bmatrix}
			1 & 0 & 0 & 0 \\
			0 & 1 & 0 & 0
		\end{bmatrix}}_{\text{estimation model}}
		\begin{bmatrix}
			p_x \\ p_y \\ v_x \\ v_y
		\end{bmatrix}+\text{noise}
	\end{equation}
	whose $\text{noise}\sim\mathcal{N}\left(0,\Sigma_z\right)$. Because our estimation includes both position and velocity, but out measurement only has position, we have an measurement matrix $C$ to translate our measurement to estimation.

	\subsection{Prediction}\label{sub:predict}
	\begin{equation}
		p_t=\mathbf{A}p_{t-1}\mathbf{A}^T+Q
	\end{equation}
	where $Q$ is the convarience matrix of $E_x$.

	\subsection{Measure}\label{sub:measure}
	The Kalman gain would be
	\begin{equation}
		k_k=p_k\mathbf{H}^T{\left(\mathbf{H}p_k\mathbf{H}^T+R\right)}^{-1}
	\end{equation}
	where $R$ is the convarience matrix of $E_z$. The system noise covariance matrix:
	\begin{equation}
		Q=
		\begin{bmatrix}
			\sigma_1^2       &                  & \sigma_1\sigma_3 &                 \\
			                 & \sigma_2^2       &                  & \sigma_2\sigma4 \\
			\sigma_3\sigma_1 &                  & \sigma_3^2       &                 \\
			                 & \sigma_4\sigma_2 &                  & \sigma_4^2      \\
		\end{bmatrix}
	\end{equation}
	has been erased off the zero elements because we have assumed $x$ and $y$ velocities are independent.
	%----------------------------------------------------------------------------------------
	%	SECTION 2
	%----------------------------------------------------------------------------------------
	\section{Program}\label{sec:program}

	\subsection{Initial Conditions}\label{sub:Initial Conditions}
	The arguments we put into the program are explained below, with their default initial input.
	\begin{equation}
		\mathtt{true\_initial\_state}=
		\begin{bmatrix}
			p_{x0} \\ p_{y0} \\ v_{x0} \\ v_{y0}
		\end{bmatrix}=
		\begin{bmatrix}
			10 \\ 10 \\ 0 \\ 0
		\end{bmatrix}
	\end{equation}

	\begin{equation}
		\mathtt{initial\_estimation}=
		\begin{bmatrix}
			p_{x0} \\ p_{y0} \\ v_{x0} \\ v_{y0}
		\end{bmatrix}=
		\begin{bmatrix}
			0 \\ 0 \\ 0 \\ 0
		\end{bmatrix}
	\end{equation}

	\begin{equation}
			\mathtt{acceleration}=
			\begin{bmatrix}
				a_{x0} \\ a_{y0}
			\end{bmatrix}=
			\begin{bmatrix}
				1 \\ 1
			\end{bmatrix}
	\end{equation}

	\begin{equation}
			(\mathtt{int})\ \mathtt{number\_of\_iters}=10
	\end{equation}

	\begin{equation}
			(\mathtt{double})\ \mathtt{delta\_t}=1.0
	\end{equation}

	\begin{equation}
			\mathtt{sig\_acceleration}=
			\begin{bmatrix}
				\mathrm{std\_dev}(a_x) & \mathrm{std\_dev}(a_y)
			\end{bmatrix}=
			\begin{bmatrix}
				0.1 & 0.1
			\end{bmatrix}
	\end{equation}

	\begin{equation}
			\mathtt{var\_obs}=
			\begin{bmatrix}
				\mathrm{var}(\mathrm{observed}\_P_x) & 0 \\
				0 & \mathrm{var}(\mathrm{observed}\_P_y) \\
			\end{bmatrix}=
			\begin{bmatrix}
				1 & 0 \\
				0 & 1
			\end{bmatrix}
	\end{equation}

	\subsection{Process}\label{sub:Process}
	The program generates the noisy observations by adding random Gaussian noise to the true trajectory.

	%----------------------------------------------------------------------------------------
	%	SECTION 3
	%----------------------------------------------------------------------------------------
	\section{Conclusion}
	We have plotted within the program with \texttt{matplotlib}.
\end{document}
